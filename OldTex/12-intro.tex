\StructuredChapter{Введение}
Несмотря на свою краткую историю киберспорт быстро обрел поклонников по всему миру. Лучше всего киберспорт развит в Корее, также он очень популярен в Америке и в Европе. Киберспорт начал очень быстро развиваться в Китае, при том, что он там появился совсем недавно. В России же наоборот киберспорт существует давно, но не настолько развит, как хотелось бы. Россия даже стала первой страной в мире, которая признала киберспорт официальным видом спорта 25 июля 2001 года. В июле 2006 г. киберспорт был исключён из Всероссийского реестра видов спорта вследствие того, что он не соответствовал критериям, необходимым для включения в этот реестр. Тем не менее, в связи с тем что киберспортивные дисциплины стали набирать популярность, а так же привлекать большие инвестиции,  7 июня 2016 года был опубликован приказ Министерства Спорта о включении Компьютерного спорта в реестр официальных видов спорта Российской Федерации \cite{prikaz}. 

Актуальность данной работы в том что киберспорт быстроразвивающийся вид спорта, который набирает популярность по всему миру, именно по этому он привлекает инвестиции, спонсоров и людей. 

\textbf{Цель работы} – реализовать базу данных 	для организации турниров по киберспортивной дисциплине “Dota 2” и их управления.

Для реализации данного проекта, необходимо решить ряд задач:
\begin{itemize}
	\item формализовать задание, определить необходимый функционал;
	\item определить роли пользователей;
	\item проанализировать существующие аналоги;
	\item проанализировать модели базы данных;
	%\item описать структуру базы данных;
	\item построить инфологическую модель базы данных;
	\item спроектировать приложение для доступа к базе данных;
	\item создать и заполнить базу данных;
	\item реализовать интерфейс для доступа к базе данных;
	\item разработать программу, реализующую поставленную задачу.
\end{itemize}