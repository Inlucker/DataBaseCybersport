\chapter{Исследовательская часть}
\label{cha:research}

В данном разделе показаны примеры работы разработанного приложения, а также проведен сравнительный анализ времени выполнения запросов к базе данных с использованием индексов и без.

На рисунках \ref{img:ex1}-\ref{img:ex3} показаны примеры работы разработанного приложения

\imgsc{H}{0.5}{ex1}{пример работы разработанного приложения, часть 1}

\imgsc{H}{0.5}{ex3}{пример работы разработанного приложения, часть 2}

%\section{Технические характеристики}

Ниже приведены технические характеристики устройства, на котором были проведены эксперименты при помощи разработанного ПО:
\begin{itemize}
	\item операционная система: Windows 10 (64-разрядная);
	\item оперативная память: 32 GB;
	\item процессор: Intel(R) Core(TM) i7-7700K CPU @ 4.20GHz;
	\item количество ядер: 4;
	\item количество потоков: 8.
\end{itemize}

\section{Сравнительный анализ времени выполнения запросов}

Чтобы провести сравнительный анализ времени выполнения запросов к базе данных с использованием индексов и без, замерялось время выполнения разных запросов 10000 раз, а затем делилось на количество замеров. Затем, аналогичным способом замерялось время выполнения тех же запросов, но с использованием индексов.

Запросы для которых проводились замеры:
\begin{itemize}
	\item найти все матчи по значению поля $tournament\_id$ (8 000 записей);
	\item найти всех игроков по значению поля $team\_id$ (5 000 записей);
	\item найти всех комментаторов по значению поля $studio\_id$ (5 000 записей).
\end{itemize}

В таблице \ref{tab:time_compare} показаны результаты анализа.

\begin{table}[H]
\caption{Время выполнения запросов с использованием индексов и без}
\label{tab:time_compare}
\centering
\resizebox{\columnwidth}{!}{
	\begin{tabular}{|c|c|c|}
		\hline
		Запрос & \specialcell{Время выполнения \\ без индексов, мс} & \specialcell{Время выполнения \\ с индексами, мс} \\
		\hline
		Поиск матчей & 0.4943498 & 0.0073635 \\
		\hline
		Поиск игроков & 0.2633266 & 0.0112785 \\
		\hline
		Поиск комментаторов & 0.2621542 & 0.0072735 \\
		\hline
\end{tabular}}
\end{table}

\section*{Вывод из иследовательской части}
В данном разделе показаны примеры работы разработанного приложения, а также был проведен анализ времени выполнения запросов к базе данных с использованием индексов и без. Как и ожидалось, время выполнения запросов с использованием индексов меньше, чем без них. Также можно подметить, что чем больше данных в таблице, тем больше разница во времени выполнения запроса к ней.