\StructuredChapter{Реферат}
%\StructuredChapter{Введение}

\noindent\textbf{Ключевые слова}: Базы Данных, SQL, PostgreSQL, Киберспорт\\

%Объектом разработки является базы данных для хранения конфигураций нейронных сетей.

\textbf{Цель работы} – реализовать базу данных 	для организации турниров по киберспортивной дисциплине “Dota 2” и их управления.

Для реализации данного проекта, необходимо решить ряд задач:
\begin{itemize}
	\item формализовать задание, определить необходимый функционал;
	\item определить роли пользователей;
	\item проанализировать существующие аналоги;
	\item проанализировать модели базы данных;
	%\item описать структуру базы данных;
	\item построить инфологическую модель базы данных;
	\item спроектировать приложение для доступа к базе данных;
	\item создать и заполнить базу данных;
	\item реализовать интерфейс для доступа к базе данных;
	\item разработать программу, реализующую поставленную задачу.
	% Добавить цель для исследовния
\end{itemize}

%В результате выполнения работы была спроектирована и разработана база данных для хранения конфигураций нейронных сетей.
%
%По результатам экспериментальных измерений, использование кэширования при получении информации из базы данных позволяет снизить времени отклика системы вплоть до 39 раз, при условии, что запрашиваемая информация находится в кэше.