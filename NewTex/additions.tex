%\addcontentsline{toc}{chapter}{ПРИЛОЖЕНИЕ А}
\StructuredChapter{ПРИЛОЖЕНИЕ А }

\appendix
%\titleformat{\section}[display]
%  {\normalfont\large\bfseries}
%  {\centering ПРИЛОЖЕНИЕ\ \thesection}
%  {14pt}{\large\centering}
%\renewcommand{\thesection}{\Asbuk{section}}

% Я не знаю, как нумеровать формулы в приложении, как по стандарту (например "(A.1)")

%\section{Поддержка SQL Server в ОС Fedora Workstation 35}

%\section{Создание таблиц}
\begin{lstinputlisting}[
	caption={создание таблиц, часть 1},
	label={TableCreation},
	language=SQL,
	linerange={1-39},
	]{./inc/src/TableCreation.sql}
\end{lstinputlisting}

\newpage
\begin{lstinputlisting}[
	caption={создание таблиц, часть 2},
	label={TableCreation},
	language=SQL,
	linerange={41-78},
	]{./inc/src/TableCreation.sql}
\end{lstinputlisting}

\newpage
\begin{lstinputlisting}[
	caption={создание таблиц, часть 3},
	label={TableCreation},
	language=SQL,
	linerange={80-115},
	]{./inc/src/TableCreation.sql}
\end{lstinputlisting}

\newpage
\begin{lstinputlisting}[
	caption={копирование сгенерированных таблиц в базу данных},
	label={Copy},
	language=SQL,
	]{./inc/src/Copy.sql}
\end{lstinputlisting}

\newpage
\begin{lstinputlisting}[
	caption={создание views, часть 1},
	label={Views},
	language=SQL,
	linerange={1-36},
	]{./inc/src/Views.sql}
\end{lstinputlisting}

\newpage
\begin{lstinputlisting}[
	caption={создание views, часть 2},
	label={Views},
	language=SQL,
	linerange={38-80},
	]{./inc/src/Views.sql}
\end{lstinputlisting}

\newpage
\begin{lstinputlisting}[
	caption={создание триггеров, часть 1},
	label={Triggers},
	language=SQL,
	linerange={1-24},
	]{./inc/src/Triggers.sql}
\end{lstinputlisting}

\newpage
\begin{lstinputlisting}[
	caption={создание триггеров, часть 2},
	label={Triggers},
	language=SQL,
	linerange={26-68},
	]{./inc/src/Triggers.sql}
\end{lstinputlisting}

\newpage
\begin{lstinputlisting}[
	caption={создание ролей},
	label={Roles},
	language=SQL,
	]{./inc/src/Roles.sql}
\end{lstinputlisting}
