\chapter{Практика на предприятии}

\section{Технологическая часть практики. Индивидуальное задание}

Индивидуальное задание, выданное на предприятии, включало следующее:
\begin{enumerate}
	\item Выдача первоначальной задачи, заключающейся в ознакомлении с отделом
	\item Знакомство с лабораторным оборудованием, программно-техническими средставами доступные сотрудникам отдела
	\item Знакомство с утилитами командной строки, компонентами, поставляемые вместе в набором свободных библиотек, позволяющими записывать, конвертироватьи передавать цифровые аудио- и видеозаписи в различных форматах -- FFmpeg:
	\begin{itemize}
		\item ffmpeg - предназначена для конвертирования видеофайла из одного формата в другой, захвата видео в реальном времени. 
		\item ffplay - медиаплеер
		\item ffprobe - предназначена для сбора и отображения информации о медиафайлах и мультимедиапотоках, доступных устройствах, кодеках, форматах и т.д.
	\end{itemize}
	\item Изучение базовых возможностей библиотеки алгоритмов компьютерного зрения и обработки изображения - OpenCV
	\item Основная задача
\end{enumerate}

	Основная задача формулируется следующим образом:

	Разработать программу, преобразующее черно-белое изображение в цветное, используя цветовую карту, которая задается пользователем. 
	Цветовая карта представляет из себя таблицу: одному значению (будем называть его условной температурой\footnote{ Реальная, но более сложная обработка изображения должна обеспечивать калибровку тепловизора, которая предполагает коррекцию получаемого изображения на основе коэффициентов излучения объектов, попавших на изображение } (УТ)) три значения в диапазоне [0; 255] --- цвет в формате BGR.
	
	Программа предназначается для работы с изображением, полученным от тепловизионной камеры, поэтому в ее функционал должна входить возможность вывода цветовой карты в виде полосы, отображающей цветовой переход от максимальной к минимальной УТ.
	Обеспечить возможность ввода диапазона УТ, имеющих важно значения для пользователя. Следует обеспечить удаление регионов изображения, которые соответствуют условным температурам, выходящим за границы диапазона УТ, задаваемого пользователем. 
	На температурном переходе следует отменить те участки, которые отбрасываются.
	Программа должна обеспечить работу с одиночным изображением, видео-потоком с камеры, видеофайлом. 

	Интерфейс программы должен быть консольным. Программа должна предусматрить возможность как немедленного вывода преобразованного изображения/видеофайла, так и его сохранение. Предусмотреть возможность вывода исходного изображения.

Исходные данные:
\begin{itemize}
	\item Путь до файла-устройства (например /dev/video0) // до файла с изображением в формате jpg, png, gif // до видеофайла формата avi, mpeg
	\item Имя файла, в который будет записан результат работы программы
	\item Файл с описанием цветовой карты. Каждая строка файла в формате "uint8 uint8 uint8 float32"
\end{itemize}
Выходные данные:
\begin{itemize}
	\item Преобразованное изображение/видео, сохраненное в требуемый файл
\end{itemize}

	Задача была решена с использованием языка программирования Python, так как требовалось написать программу в сжатые сроки. Написание программы заняло 1.5 рабочих дня. Листинг ключевых частей программы приведен в приложении B. В приложении Г представлены результаты работы програмы при корректном задании исходным параметров пользователем. 

	Проблемы, которые возникли при решении задачи:
\begin{itemize}
	\item Используемый ЯП (Python) не позволил быстро выводить картинку (обработка одного изображения занимает значительное время, не позволяющее использовать программу для обработки данных в реальном масштабе времени)
	\item Уже на завершающей стадии разработки, т.е. во время отладки программы, была выявлена серьезная ошибка, совершенная на этапе проектирования: удаление регионов изображения выполняется не всегда правильно: процесс удаления регионов использует исходное изображение, в то время как интенсивности цветов результирующего изображения нелинейно зависит от исходных интенсивностей. Это ведет к тому, что появляются некоторые участки регионов, которые не должны появляться. Проявление данной проблемы представлено в приложении Г
\end{itemize}

	Стоит отметить, что тестирование программы выполнялось с использованием лабораторного оборудования, предоставленного АО "РПКБ": тепловизор "Plug 417 ASIC" и преобразователь аналогово сигнала "DVD EZMaker 7 - C039".

\section{Вывод}
	В ходе работы над индивидуальным заданием были получены навыки работы со средствами обработки изображения, позволяющими конвертировать, кодировать и декодировать, выполнять фильтрацию медиафайлов: видеофайлов и изображений. Прохождение практики на предприятии позволило поработать с таким лабораторным оборудованием, как тепловизор, который может быть установлен на беспилотную летающую платформу для выполнения спец. работ, требующих, кроме того, специального программного обеспечения. Также в ходе работы по конструированию ПО в рамках индивидуального задания были получены некоторые навыки работы в команде со специалистами в области практического применения методов компьютерного зрения, разработчиками ПО для встроенных систем.

	Полученные навыки работы со средствами обработки изображения должны помочь мне в ходе работы по дальнейшей разработке программы в рамках курсового проекта.
