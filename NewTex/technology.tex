\chapter{Технологическая часть}
\label{cha:impl}

В данном разделе приведен код. Код создания таблиц смотри в Приложении А.
%Много много много много много много много много много много много много много много много много много много много много много много много много много много много много много много текста. Много много много много много много много много много много много много много много много много много много много много много много много много много много много много много много текста.

\section{Обзор СУБД}

В данном подразделе будут рассмотрены популярные СУБД, которые могут быть использованы для реализации хранения данных в разрабатываемом программном продукте.\\

\noindent\textbf{Oracle Database}\\

Oracle Database~\cite{oracle} -- объектно-реляционная система управления базами данных разрабатываемая компанией Oracle \cite{oracle-company}. На данный момент, рассматриваемая СУБД является наиболее популярной в мире \cite{oracle-popular}. 

Все транзакции Oracle Database обладают свойствами ACID~\cite{acid}, поддерживает триггеры, внешние ключи и хранимые процедуры. Данная СУБД подходит для разнообразных рабочих нагрузок и может использоваться практически в любых задачах. Особенностью Oracle Database является быстрая работа с большими массивами данных.\\

\noindent\textbf{MySQL}\\

MySQL~\cite{mysql} -- свободная реляционная система управления базами данных. Разработку и поддержку MySQL осуществляет корпорация Oracle.

Рассматриваемая СУБД имеет два основных движка хранения данных: InnoDB~\cite{innodb} и myISAM~\cite{myisam}. Движок InnoDB полностью совместим с принципами ACID, в отличии от движка myISAM. СУБД MySQL подходит для использования при разработке веб-приложений, что объясняется очень тесной интеграцией с популярными языками PHP \cite{php} и Perl \cite{perl}.

\newpage

\noindent\textbf{PostgreSQL}\\

PostgreSQL~\cite{postgresql} -- это свободно распространяемая объектно-реляционная система управления базами данных, наиболее развитая из открытых СУБД в мире и являющаяся реальной альтернативой коммерческим базам данных \cite{postgresql-fact}.

PostgreSQL предоставляет транзакции, обладающие свойствами ACID, автоматически обновляемые представления, материализованные представления, триггеры, внешние ключи и хранимые процедуры. Данная СУБД предназначена для обработки ряда рабочих нагрузок, от отдельных компьютеров до хранилищ данных или веб-сервисов с множеством одновременных пользователей. 

