\chapter{Технологическая часть}
\label{cha:impl}

В данном разделе проанализированы и выбраны: язык программирования, инструменты для создания пользовательского интерфейса, выбрана среда разработки и другие инструменты для реализации проекта, произведен анализ существующих СУБД и на основе него сделан выбор одной из них, а также произведено верхнеуровневое разбиение на компоненты и описаны детали реализации desktop приложения и базы данных.
%Много много много много много много много много много много много много много много много много много много много много много много много много много много много много много много текста. Много много много много много много много много много много много много много много много много много много много много много много много много много много много много много много текста.

\section{Основные инструменты, используемые для реализации}

\noindent\textbf{Язык программирования C++.}

C++ -- язык программирования общего назначения с уклоном в сторону системного программирования \cite{Cpp}

Данный язык, достаточно популярен и широко распространен, кроме этого, он имеет ряд плюсов, описанных ниже:

\begin{itemize}
	\item высокая производительность: язык спроектирован так, чтобы дать программисту максимальный контроль над всеми аспектами структуры и порядка исполнения программы; один из базовых принципов C++ «не платишь за то, что не используешь» то есть ни одна из языковых возможностей, приводящая к дополнительным накладным расходам, не является обязательной для использования; имеется возможность работы с памятью на низком уровне;
	\item кроссплатформенность: стандарт языка C++ накладывает минимальные требования на ЭВМ для запуска скомпилированных программ;
	\item поддержка различных стилей программирования: традиционное императивное программирование (структурное, объектно-ориентированное), обобщенное программирование, функциональное программирование, порождающее метапрограммирование.
\end{itemize}

Исходя из вышеперечисленных плюсов, очевиден выбор данного языка программирования для реализации поставленной задачи.

\noindent\textbf{Кроссплатформенный фреймворк Qt.}

Для реализации данного проекта, необходима была библиотека, упрощающая создание интерфейса. На примете были Qt и MFC. Чтобы сделать выбор, пришлось их сравнить. 

Qt – кроссплатформенный фреймворк для разработки программного обеспечения на языке программирования C++. Есть также «привязки» ко многим другим языкам программирования: Python — PyQt, PySide; Ruby — QtRuby; Java — Qt Jambi; PHP — PHP-Qt и другие. Поддерживаемые платформы включают Linux, OS X, Windows, VxWorks, QNX, Android, iOS, BlackBerry, ОС Sailfish и другие \cite{Qt}.

Qt позволяет запускать написанное с его помощью программное обеспечение в большинстве современных операционных систем путем простой компиляции программы для каждой системы без изменения исходного кода (кроссплатформенность). Включает в себя все основные классы, которые могут потребоваться при разработке прикладного программного обеспечения, начиная от элементов графического интерфейса и заканчивая классами для работы с сетью, базами данных и XML. Является полностью объектно-ориентированным, расширяемым и поддерживающим технику компонентного программирования.

Комплектуется визуальной средой разработки графического интерфейса Qt Designer, позволяющей создавать диалоги и формы.

Также существует возможность расширения привычной функциональности виджетов, связанной с размещением их на экране, отображением, перерисовкой при изменении размеров окна.

Мета-объектная система — часть ядра фреймворка для поддержки в С++ таких возможностей, как сигналы и слоты для коммуникации между объектами в режиме реального времени и динамических свойств системы.

Одним из преимуществ проекта Qt является наличие качественной документации. Статьи документации снабжены большим количеством примеров. Исходный код самой библиотеки хорошо форматирован, подробно комментирован, что также упрощает изучение Qt.

Отличительная особенность — использование мета-объектного компилятора — предварительной системы обработки исходного кода. Расширение возможностей обеспечивается системой плагинов, которые возможно размещать непосредственно в панели визуального редактора. Но минусом получается то, что код написанный с помощью Qt нельзя скомпилировать на другом компьютере без установки фреймворка.

Microsoft Foundation Classes – библиотека на языке C++, разработанная Microsoft и призванная облегчить разработку GUI-приложений для Microsoft Windows путём использования богатого набора библиотечных классов.
Во-первых, если сравнивать только работу с GUI, то данная библиотека работает только под Windows, то есть ни о какой кроссплатформенности речи не идёт. Но не стоит забывать о том, что Qt в отличии от MFC имеет множество других полезных классов. Во-вторых, если же в MFC создать каркас приложения без дизайнера достаточно сложно, то в Qt это зачастую даже намного удобнее и проще.

Поскольку функционал Qt намного шире, то для реализации проекта был выбран именно фреймворк Qt.

\noindent\textbf{Среда разработки Qt creator.}

Qt Creator (ранее известная под кодовым названием Greenhouse) — кроссплатформенная свободная IDE для языков С, С++ и QML. Разработана Trolltech (Digia) для работы с фреймворком Qt. Включает в себя графический интерфейс отладчика и визуальные средства разработки интерфейса как с использованием QtWidgets, так и QML. Поддерживаемые компиляторы: GCC, Clang, MinGW, MSVC, Linux ICC, GCCE, RVCT, WINSCW.

Основная задача Qt Creator — упростить разработку приложения с помощью фреймворка Qt на разных платформах. Поэтому для работы с данной библиотекой был выбран именно он.

\noindent\textbf{Система версионного контроля git.}

Для хранения исходников используется система Git (на портале github.com), т.к. это крупнейший веб-сервис для хостинга IT-проектов и их совместной разработки.

\section{Выбор СУБД}

В данном подразделе произведен анализ популярных СУБД, которые могут быть использованы для реализации хранения данных в разрабатываемом программном продукте.\\

\noindent\textbf{Oracle Database.}

Oracle Database~\cite{oracle} -- объектно-реляционная система управления базами данных разрабатываемая компанией Oracle \cite{oracle-company}. На данный момент, рассматриваемая СУБД является наиболее популярной в мире \cite{oracle-popular}. 

Все транзакции Oracle Database обладают свойствами ACID~\cite{acid}, поддерживает триггеры, внешние ключи и хранимые процедуры. Данная СУБД подходит для разнообразных рабочих нагрузок и может использоваться практически в любых задачах. Особенностью Oracle Database является быстрая работа с большими массивами данных.\\

\noindent\textbf{MySQL.}

MySQL~\cite{mysql} -- свободная реляционная система управления базами данных. Разработку и поддержку MySQL осуществляет корпорация Oracle.

Рассматриваемая СУБД имеет два основных движка хранения данных: InnoDB~\cite{innodb} и myISAM~\cite{myisam}. Движок InnoDB полностью совместим с принципами ACID, в отличии от движка myISAM. СУБД MySQL подходит для использования при разработке веб-приложений, что объясняется очень тесной интеграцией с популярными языками PHP \cite{php} и Perl \cite{perl}.\\

\noindent\textbf{PostgreSQL.}

PostgreSQL~\cite{postgresql} -- это свободно распространяемая объектно-реляционная система управления базами данных, наиболее развитая из открытых СУБД в мире и являющаяся реальной альтернативой коммерческим базам данных \cite{postgresql-fact}.

PostgreSQL предоставляет транзакции, обладающие свойствами ACID, автоматически обновляемые представления, материализованные представления, триггеры, внешние ключи и хранимые процедуры. Данная СУБД предназначена для обработки ряда рабочих нагрузок, от отдельных компьютеров до хранилищ данных или веб-сервисов с множеством одновременных пользователей. 

Для реализации проекта была выбрана СУБД postgreSQL, потому что она поддерживает язык \texttt{plpython3u}, обладает мощными и надёжными механизмами транзакций и репликации, легко расширяема, а также проста в использовании и развертывании. Для взаимодействия приложения с базой данных была выбрана библиотека libpq \cite{libpq}.

\section{Реализация}
Для реализации desktop приложения было произведено верхнеуровневое разбиение на компоненты. На уровне пользовательского интерфейса были выделены компоненты BaseWindow, TeamWindow, StudioWindow и TournamentWindow каждый из которых отвечает за взаимодействие приложения с пользователями соответствующих ролей. Компоненты бизнес логики были разбиты аналогично: контроллеры реализуют взаимодействие между соответствующим ему пользовательским интерфейсом и приложением. Кроме этого, необходимо реализовать компоненты доступа к данным (репозитории) для каждой таблицы, которые будут \textquotedblleft мостом\textquotedblright \space между приложением и базой данных.

Также необходимо реализовать саму базу данных: создание таблиц, views, триггеров и ролей. В приложении А приведены листинга кода реализующих базу данных.

\section*{Вывод из технологической части}
В данном разделе были проанализированы и выбраны основные инструменты для реализации проекта, а также произведен анализ существующих СУБД, на основе которого была выбрана СУБД postgreSQL.