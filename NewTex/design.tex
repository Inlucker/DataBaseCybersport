\chapter{Конструкторская часть}
\label{cha:design}

В данном разделе приведены таблицы, инфологическая модель и схемы триггеров базы данных.

\section{Таблицы базы данных}
Исходя из приведенных выше use-case и ER диаграмм можно выделить следующие таблицы:

\begin{itemize}
	\item таблица ролей пользователей;
	\item таблица пользователей;
	\item таблица стран;
	\item таблица спонсоров;
	\item таблица команд;
	\item табилца игроков;
	\item таблица студий;
	\item таблица комментаторов;
	\item таблица турниров;
	\item таблица турниров и участвующих в них команд;
	\item таблица матчей.
\end{itemize}

\newpage
\section{Инфологическая модель базы данных}
Ниже, на рисунке \ref{img:InfoModelDB.pdf} приведена инфологическая модель базы данных, определяющая связи между полями будущих таблиц.

\imgsc{H}{0.8}{InfoModelDB.pdf}{Инфологическая модель базы данных}

\newpage
\section{Схемы триггеров}
На рисунке \ref{img:tr_whenDeleteTeamCaptain} изображена схема триггера который срабатывает при удаление пользователя с ролью \textquotedblleft Team captain\textquotedblright.

\imgsc{H}{1}{tr_whenDeleteTeamCaptain}{схема триггера $tr\_whenDeleteTeamCaptain$}

На рисунке \ref{img:tr_whenDeleteStudioOwner} изображена схема триггера который срабатывает при удаление пользователя с ролью \textquotedblleft Studio owner\textquotedblright.

\imgsc{H}{1}{tr_whenDeleteStudioOwner}{схема триггера $tr\_whenDeleteStudioOwner$}

На рисунке \ref{img:tr_whenDeleteTournamentOrganizer} изображена схема триггера который срабатывает при удаление пользователя с ролью \textquotedblleft Tournament organizer\textquotedblright.

\imgsc{H}{1}{tr_whenDeleteTournamentOrganizer}{схема триггера $tr\_whenDeleteTournamentOrganizer$}

\section*{Вывод из конструкторской части}
В данном разделе были выделены таблицы, разработна инфологическая модель и  схемы триггеров базы данных.