\usepackage{csvsimple}

%\addbibresource{biblio.bib}
%\bibliography{biblio}
\lstset{frame=none, tabsize=4}

%ADDED
\newcommand{\specialcell}[2][c]
{
	\begin{tabular}[#1]{@{}l@{}}#2\end{tabular}
}

\usepackage{listings}

% Значения по умолчанию
\lstset{
	basicstyle= \footnotesize,
	breakatwhitespace=true,% разрыв строк только на whitespacce
	breaklines=true,       % переносить длинные строки
	%   captionpos=b,          % подписи снизу -- вроде не надо
	inputencoding=koi8-r,
	numbers=none,          % нумерация слева
	numberstyle=\footnotesize,
	showspaces=false,      % показывать пробелы подчеркиваниями -- идиотизм 70-х годов
	showstringspaces=false,
	showtabs=false,        % и табы тоже
	stepnumber=1,
	tabsize=4,              % кому нужны табы по 8 символов?
	frame=single
}

%FOR SQL
\definecolor{ltgreen}{rgb}{0,0.5,0}
\definecolor{ltgray}{rgb}{0.5,0.5,0.5}
\definecolor{dkred}{rgb}{0.5,0,0}

\lstset{
	commentstyle=\color{ltgray},
	keywordstyle=\color{dkred},
	stringstyle=\color{ltgreen},
	language=SQL
}

%FOR GOST BIBLIO
%\usepackage[T2A]{fontenc} % Поддержка русских букв
%\usepackage[utf8]{inputenc} % Кодировка utf8
%\usepackage[english, russian]{babel} % Языки: русский, английский
%\usepackage{pscyr} % Нормальные шрифты
\addto\captionsrussian{\def\refname{Список используемой литературы}}